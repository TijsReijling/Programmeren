% Options for packages loaded elsewhere
\PassOptionsToPackage{unicode}{hyperref}
\PassOptionsToPackage{hyphens}{url}
%
\documentclass[
]{article}
\usepackage{amsmath,amssymb}
\usepackage{iftex}
\ifPDFTeX
  \usepackage[T1]{fontenc}
  \usepackage[utf8]{inputenc}
  \usepackage{textcomp} % provide euro and other symbols
\else % if luatex or xetex
  \usepackage{unicode-math} % this also loads fontspec
  \defaultfontfeatures{Scale=MatchLowercase}
  \defaultfontfeatures[\rmfamily]{Ligatures=TeX,Scale=1}
\fi
\usepackage{lmodern}
\ifPDFTeX\else
  % xetex/luatex font selection
\fi
% Use upquote if available, for straight quotes in verbatim environments
\IfFileExists{upquote.sty}{\usepackage{upquote}}{}
\IfFileExists{microtype.sty}{% use microtype if available
  \usepackage[]{microtype}
  \UseMicrotypeSet[protrusion]{basicmath} % disable protrusion for tt fonts
}{}
\makeatletter
\@ifundefined{KOMAClassName}{% if non-KOMA class
  \IfFileExists{parskip.sty}{%
    \usepackage{parskip}
  }{% else
    \setlength{\parindent}{0pt}
    \setlength{\parskip}{6pt plus 2pt minus 1pt}}
}{% if KOMA class
  \KOMAoptions{parskip=half}}
\makeatother
\usepackage{xcolor}
\usepackage[margin=1in]{geometry}
\usepackage{color}
\usepackage{fancyvrb}
\newcommand{\VerbBar}{|}
\newcommand{\VERB}{\Verb[commandchars=\\\{\}]}
\DefineVerbatimEnvironment{Highlighting}{Verbatim}{commandchars=\\\{\}}
% Add ',fontsize=\small' for more characters per line
\usepackage{framed}
\definecolor{shadecolor}{RGB}{248,248,248}
\newenvironment{Shaded}{\begin{snugshade}}{\end{snugshade}}
\newcommand{\AlertTok}[1]{\textcolor[rgb]{0.94,0.16,0.16}{#1}}
\newcommand{\AnnotationTok}[1]{\textcolor[rgb]{0.56,0.35,0.01}{\textbf{\textit{#1}}}}
\newcommand{\AttributeTok}[1]{\textcolor[rgb]{0.13,0.29,0.53}{#1}}
\newcommand{\BaseNTok}[1]{\textcolor[rgb]{0.00,0.00,0.81}{#1}}
\newcommand{\BuiltInTok}[1]{#1}
\newcommand{\CharTok}[1]{\textcolor[rgb]{0.31,0.60,0.02}{#1}}
\newcommand{\CommentTok}[1]{\textcolor[rgb]{0.56,0.35,0.01}{\textit{#1}}}
\newcommand{\CommentVarTok}[1]{\textcolor[rgb]{0.56,0.35,0.01}{\textbf{\textit{#1}}}}
\newcommand{\ConstantTok}[1]{\textcolor[rgb]{0.56,0.35,0.01}{#1}}
\newcommand{\ControlFlowTok}[1]{\textcolor[rgb]{0.13,0.29,0.53}{\textbf{#1}}}
\newcommand{\DataTypeTok}[1]{\textcolor[rgb]{0.13,0.29,0.53}{#1}}
\newcommand{\DecValTok}[1]{\textcolor[rgb]{0.00,0.00,0.81}{#1}}
\newcommand{\DocumentationTok}[1]{\textcolor[rgb]{0.56,0.35,0.01}{\textbf{\textit{#1}}}}
\newcommand{\ErrorTok}[1]{\textcolor[rgb]{0.64,0.00,0.00}{\textbf{#1}}}
\newcommand{\ExtensionTok}[1]{#1}
\newcommand{\FloatTok}[1]{\textcolor[rgb]{0.00,0.00,0.81}{#1}}
\newcommand{\FunctionTok}[1]{\textcolor[rgb]{0.13,0.29,0.53}{\textbf{#1}}}
\newcommand{\ImportTok}[1]{#1}
\newcommand{\InformationTok}[1]{\textcolor[rgb]{0.56,0.35,0.01}{\textbf{\textit{#1}}}}
\newcommand{\KeywordTok}[1]{\textcolor[rgb]{0.13,0.29,0.53}{\textbf{#1}}}
\newcommand{\NormalTok}[1]{#1}
\newcommand{\OperatorTok}[1]{\textcolor[rgb]{0.81,0.36,0.00}{\textbf{#1}}}
\newcommand{\OtherTok}[1]{\textcolor[rgb]{0.56,0.35,0.01}{#1}}
\newcommand{\PreprocessorTok}[1]{\textcolor[rgb]{0.56,0.35,0.01}{\textit{#1}}}
\newcommand{\RegionMarkerTok}[1]{#1}
\newcommand{\SpecialCharTok}[1]{\textcolor[rgb]{0.81,0.36,0.00}{\textbf{#1}}}
\newcommand{\SpecialStringTok}[1]{\textcolor[rgb]{0.31,0.60,0.02}{#1}}
\newcommand{\StringTok}[1]{\textcolor[rgb]{0.31,0.60,0.02}{#1}}
\newcommand{\VariableTok}[1]{\textcolor[rgb]{0.00,0.00,0.00}{#1}}
\newcommand{\VerbatimStringTok}[1]{\textcolor[rgb]{0.31,0.60,0.02}{#1}}
\newcommand{\WarningTok}[1]{\textcolor[rgb]{0.56,0.35,0.01}{\textbf{\textit{#1}}}}
\usepackage{graphicx}
\makeatletter
\newsavebox\pandoc@box
\newcommand*\pandocbounded[1]{% scales image to fit in text height/width
  \sbox\pandoc@box{#1}%
  \Gscale@div\@tempa{\textheight}{\dimexpr\ht\pandoc@box+\dp\pandoc@box\relax}%
  \Gscale@div\@tempb{\linewidth}{\wd\pandoc@box}%
  \ifdim\@tempb\p@<\@tempa\p@\let\@tempa\@tempb\fi% select the smaller of both
  \ifdim\@tempa\p@<\p@\scalebox{\@tempa}{\usebox\pandoc@box}%
  \else\usebox{\pandoc@box}%
  \fi%
}
% Set default figure placement to htbp
\def\fps@figure{htbp}
\makeatother
\setlength{\emergencystretch}{3em} % prevent overfull lines
\providecommand{\tightlist}{%
  \setlength{\itemsep}{0pt}\setlength{\parskip}{0pt}}
\setcounter{secnumdepth}{-\maxdimen} % remove section numbering
\usepackage{bookmark}
\IfFileExists{xurl.sty}{\usepackage{xurl}}{} % add URL line breaks if available
\urlstyle{same}
\hypersetup{
  pdftitle={Template},
  pdfauthor={Studentnames and studentnumbers here},
  hidelinks,
  pdfcreator={LaTeX via pandoc}}

\title{Template}
\author{Studentnames and studentnumbers here}
\date{2025-06-19}

\begin{document}
\maketitle

\section{Set-up your environment}\label{set-up-your-environment}

\begin{Shaded}
\begin{Highlighting}[]
\FunctionTok{library}\NormalTok{(tidyverse)}
\end{Highlighting}
\end{Shaded}

\begin{verbatim}
## -- Attaching core tidyverse packages ------------------------ tidyverse 2.0.0 --
## v dplyr     1.1.4     v readr     2.1.5
## v forcats   1.0.0     v stringr   1.5.1
## v ggplot2   3.5.2     v tibble    3.2.1
## v lubridate 1.9.4     v tidyr     1.3.1
## v purrr     1.0.4     
## -- Conflicts ------------------------------------------ tidyverse_conflicts() --
## x dplyr::filter() masks stats::filter()
## x dplyr::lag()    masks stats::lag()
## i Use the conflicted package (<http://conflicted.r-lib.org/>) to force all conflicts to become errors
\end{verbatim}

\begin{Shaded}
\begin{Highlighting}[]
\FunctionTok{library}\NormalTok{(rmarkdown)}
\FunctionTok{library}\NormalTok{(yaml)}
\FunctionTok{library}\NormalTok{(dplyr)}
\FunctionTok{library}\NormalTok{(cbsodataR)}
\FunctionTok{library}\NormalTok{(sf)}
\end{Highlighting}
\end{Shaded}

\begin{verbatim}
## Linking to GEOS 3.13.1, GDAL 3.11.0, PROJ 9.6.0; sf_use_s2() is TRUE
\end{verbatim}

\begin{Shaded}
\begin{Highlighting}[]
\FunctionTok{library}\NormalTok{(readr)}
\end{Highlighting}
\end{Shaded}

\section{Title Page}\label{title-page}

Include your names

Tutorial group 4

C. Schouwenaar

\section{Part 1 - Identify a Social
Problem}\label{part-1---identify-a-social-problem}

Use APA referencing throughout your document.
\href{https://www.mendeley.com/guides/apa-citation-guide/}{Here's a link
to some explanation.}

\subsection{1.1 Describe the Social
Problem}\label{describe-the-social-problem}

Topic: How does income inequality contribute to differences in health
and life expectancy between high- and low-income groups in the
Netherlands?

In the Netherlands, healthcare is universally granted to all citizens
through mandatory health insurance. The government promotes healthcare
equality through regulations that ensures equal access to services by
prohibiting insurance companies to decline clients. Yet, low-income
groups in the Netherlands still experience worse health outcomes than
their wealthier counterparts.

Research by the Ministry of Health, Welfare and Sport shows that college
or university-educated people, which correlates with a higher income,
consistently score higher on health outcomes. VZinfo found that people
with primary or VMBO-level education were significantly more likely to
smoke, have obesity, and rate their health as ``poor'' compared to their
higher educated peers (VZinfo, 2023).

Institutions like the CBS document imbalances between income level and
life expectancy. Their analyses show that the wealthiest group in
society lives, on average, eight years longer and spend 25 more years in
good health than those in the lowest-income group. (Centraal Bureau voor
de Statistiek {[}CBS{]}, 2022).

Pharos provides us with more disparities by emphasising socio-economic
differences in health. According to Pharos, receiving welfare benefits
correlates with poorer health. They state that with each step up on the
social ladder, their chance of good health increases (Pharos, 2022).
These health differences across different income groups undermine social
cohesion, which makes health inequality based on ones health a social
problem that needs to be examined.

Why is this relevant? Wealth inequality in the Netherlands is growing,
making these differences even more apparent. Factors like increasing
housing costs, job security and education gaps contribute to the
widening differences between low- and high-income individuals.

This debate about health inequality isn't new. Some policy makers have
been proposing to cut deductibles (eigen risico) since they contribute
to people with lower incomes delaying their care. We aim to analyse this
inequality to quantify how large these differences are and identify
their potential causes.

What has not been examined extensively is whether this inequality holds
true across Dutch municipalities. Our aim is to analyse if this
inequality hold true throughout the Netherlands, and identify regional
differences in the relationship between wealth and health.

\section{Part 2 - Data Sourcing}\label{part-2---data-sourcing}

\subsection{2.1 Load in the data}\label{load-in-the-data}

Preferably from a URL, but if not, make sure to download the data and
store it in a shared location that you can load the data in from. Do not
store the data in a folder you include in the Github repository!

\begin{Shaded}
\begin{Highlighting}[]
\FunctionTok{setwd}\NormalTok{(}\StringTok{"\textasciitilde{}/GitHub/Programmeren/data"}\NormalTok{)}

\NormalTok{Welzijn\_Goed }\OtherTok{\textless{}{-}} \FunctionTok{read\_csv}\NormalTok{(}\StringTok{"Welzijn.goed.csv"}\NormalTok{)}
\end{Highlighting}
\end{Shaded}

\begin{verbatim}
## Rows: 900 Columns: 69
## -- Column specification --------------------------------------------------------
## Delimiter: ","
## chr (16): Kenmerken, Marges, Perioden, ScoreTevredenheidMetWerk_13, Ontevred...
## dbl (53): ID, ScoreGeluk_1, Ongelukkig_2, NietGelukkigNietOngelukkig_3, Gelu...
## 
## i Use `spec()` to retrieve the full column specification for this data.
## i Specify the column types or set `show_col_types = FALSE` to quiet this message.
\end{verbatim}

\begin{Shaded}
\begin{Highlighting}[]
\NormalTok{Ervaren\_Gezondheid }\OtherTok{\textless{}{-}} \FunctionTok{read\_csv}\NormalTok{(}\StringTok{"Ervaren\_Gezondheid.csv"}\NormalTok{)}
\end{Highlighting}
\end{Shaded}

\begin{verbatim}
## Rows: 2082 Columns: 9
## -- Column specification --------------------------------------------------------
## Delimiter: ","
## chr (7): Marges, WijkenEnBuurten, Perioden, Gemeentenaam_1, SoortRegio_2, Co...
## dbl (2): ID, Leeftijd
## 
## i Use `spec()` to retrieve the full column specification for this data.
## i Specify the column types or set `show_col_types = FALSE` to quiet this message.
\end{verbatim}

\begin{Shaded}
\begin{Highlighting}[]
\NormalTok{Levensverwachting }\OtherTok{\textless{}{-}} \FunctionTok{read\_csv}\NormalTok{(}\StringTok{"Levensverwachting.csv"}\NormalTok{)}
\end{Highlighting}
\end{Shaded}

\begin{verbatim}
## Rows: 8100 Columns: 11
## -- Column specification --------------------------------------------------------
## Delimiter: ","
## chr (6): Geslacht, Marges, Perioden, LVZonderLichamelijkeBeperkingen_3, LVZo...
## dbl (5): ID, LeeftijdOp31December, InkomenEnWelvaart, Levensverwachting_1, L...
## 
## i Use `spec()` to retrieve the full column specification for this data.
## i Specify the column types or set `show_col_types = FALSE` to quiet this message.
\end{verbatim}

\begin{Shaded}
\begin{Highlighting}[]
\NormalTok{Ervaren\_gezondheid\_wijk }\OtherTok{\textless{}{-}} \FunctionTok{read\_delim}\NormalTok{(}\StringTok{"Ervarengezondheid\_Wijk\&Buurt.csv"}\NormalTok{, }\AttributeTok{delim =} \StringTok{";"}\NormalTok{)}
\end{Highlighting}
\end{Shaded}

\begin{verbatim}
## Rows: 18003 Columns: 9
## -- Column specification --------------------------------------------------------
## Delimiter: ";"
## chr (7): Marges, WijkenEnBuurten, Perioden, Gemeentenaam_1, SoortRegio_2, Co...
## dbl (2): ID, Leeftijd
## 
## i Use `spec()` to retrieve the full column specification for this data.
## i Specify the column types or set `show_col_types = FALSE` to quiet this message.
\end{verbatim}

\begin{Shaded}
\begin{Highlighting}[]
\NormalTok{Inkomen\_per\_gemeente }\OtherTok{\textless{}{-}} \FunctionTok{read\_delim}\NormalTok{(}\StringTok{"Inkomen\_gemeente.csv"}\NormalTok{, }\AttributeTok{delim =} \StringTok{";"}\NormalTok{)}
\end{Highlighting}
\end{Shaded}

\begin{verbatim}
## Rows: 3694 Columns: 6
## -- Column specification --------------------------------------------------------
## Delimiter: ";"
## chr (6): Gemeentecode, Wijkcode, Regionaam, Gemiddeld, 40% huishoudens, 20% ...
## 
## i Use `spec()` to retrieve the full column specification for this data.
## i Specify the column types or set `show_col_types = FALSE` to quiet this message.
\end{verbatim}

\begin{Shaded}
\begin{Highlighting}[]
\NormalTok{Levensverwachting\_Gemeente }\OtherTok{\textless{}{-}} \FunctionTok{read\_delim}\NormalTok{(}\StringTok{"Levensverwacht\_Gemeente\_Wijk\&Buurt.csv"}\NormalTok{, }\AttributeTok{delim =} \StringTok{";"}\NormalTok{)}
\end{Highlighting}
\end{Shaded}

\begin{verbatim}
## Rows: 454104 Columns: 7
## -- Column specification --------------------------------------------------------
## Delimiter: ";"
## chr (5): Geslacht, Marges, RegioS, Perioden, Levensverwachting_1
## dbl (2): ID, Leeftijd
## 
## i Use `spec()` to retrieve the full column specification for this data.
## i Specify the column types or set `show_col_types = FALSE` to quiet this message.
\end{verbatim}

\begin{Shaded}
\begin{Highlighting}[]
\NormalTok{ErvarenGezondheidNL }\OtherTok{\textless{}{-}} \FunctionTok{read\_csv}\NormalTok{(}\StringTok{"ErvarenGezondheidNL.csv"}\NormalTok{)}
\end{Highlighting}
\end{Shaded}

\begin{verbatim}
## Rows: 3 Columns: 6
## -- Column specification --------------------------------------------------------
## Delimiter: ","
## chr (3): Marges, WijkenEnBuurten, Perioden
## dbl (3): ID, Leeftijd, ErvarenGezondheidGoedZeerGoed_4
## 
## i Use `spec()` to retrieve the full column specification for this data.
## i Specify the column types or set `show_col_types = FALSE` to quiet this message.
\end{verbatim}

\begin{Shaded}
\begin{Highlighting}[]
\NormalTok{gemeentegrenzen }\OtherTok{\textless{}{-}} \FunctionTok{st\_read}\NormalTok{(}\StringTok{"https://service.pdok.nl/cbs/gebiedsindelingen/2023/wfs/v1\_0?request=GetFeature\&service=WFS\&version=2.0.0\&typeName=gemeente\_gegeneraliseerd\&outputFormat=json"}\NormalTok{)}
\end{Highlighting}
\end{Shaded}

\begin{verbatim}
## Reading layer `gemeente_gegeneraliseerd' from data source 
##   `https://service.pdok.nl/cbs/gebiedsindelingen/2023/wfs/v1_0?request=GetFeature&service=WFS&version=2.0.0&typeName=gemeente_gegeneraliseerd&outputFormat=json' 
##   using driver `GeoJSON'
## Simple feature collection with 342 features and 5 fields
## Geometry type: MULTIPOLYGON
## Dimension:     XY
## Bounding box:  xmin: 13565.4 ymin: 306846.2 xmax: 278026.1 ymax: 619231.6
## Projected CRS: Amersfoort / RD New
\end{verbatim}

midwest is an example dataset included in the tidyverse package

\subsection{2.2 Provide a short summary of the
dataset(s)}\label{provide-a-short-summary-of-the-datasets}

\begin{Shaded}
\begin{Highlighting}[]
\FunctionTok{head}\NormalTok{(Welzijn\_Goed)}
\end{Highlighting}
\end{Shaded}

\begin{verbatim}
## # A tibble: 6 x 69
##      ID Kenmerken Marges  Perioden ScoreGeluk_1 Ongelukkig_2
##   <dbl> <chr>     <chr>   <chr>           <dbl>        <dbl>
## 1     0 T009002   MW00000 2013JJ00          7.7          2.5
## 2     1 T009002   MW00000 2014JJ00          7.7          2.4
## 3     2 T009002   MW00000 2015JJ00          7.7          2.8
## 4     3 T009002   MW00000 2016JJ00          7.7          2.6
## 5     4 T009002   MW00000 2017JJ00          7.7          2.8
## 6     5 T009002   MW00000 2018JJ00          7.7          2.8
## # i 63 more variables: NietGelukkigNietOngelukkig_3 <dbl>, Gelukkig_4 <dbl>,
## #   ScoreTevredenheidMetHetLeven_5 <dbl>, Ontevreden_6 <dbl>,
## #   NietTevredenNietOntevreden_7 <dbl>, Tevreden_8 <dbl>,
## #   ScoreTevredenheidOpleidingskansen_9 <dbl>, Ontevreden_10 <dbl>,
## #   NietTevredenNietOntevreden_11 <dbl>, Tevreden_12 <dbl>,
## #   ScoreTevredenheidMetWerk_13 <chr>, Ontevreden_14 <chr>,
## #   NietTevredenNietOntevreden_15 <chr>, Tevreden_16 <chr>, ...
\end{verbatim}

This data set contains and presents figures on the wellness/well-being
of the population of the Netherlands aged 18 years or older. This
factors in happiness and satisfaction with life, satisfaction with
education, work, travel, daily activities, weight, financial situation,
housing, living environment social life and total amount of time spent
on leisure. In addition, concerns about financial future, feelings of
insecurity and trust in others are included. This data can be
categorized by gender, age, education level and origin.

\href{https://opendata.cbs.nl/statline/portal.html?_la=nl&_catalog=CBS&tableId=85542NED&_theme=178}{URL
of the data download page} + Metadata

\begin{Shaded}
\begin{Highlighting}[]
\FunctionTok{head}\NormalTok{(}\StringTok{"Ervaren\_Gezondheid.csv"}\NormalTok{)}
\end{Highlighting}
\end{Shaded}

\begin{verbatim}
## [1] "Ervaren_Gezondheid.csv"
\end{verbatim}

The table contains estimated percentages of indicators related to
health, social situation, and lifestyle at neighborhood, district, and
municipal levels, based on a conducted survey.

\href{https://statline.rivm.nl/portal.html?_la=nl&_catalog=RIVM&tableId=50120NED&_theme=94}{URL
of the data download page} + Metadata

\begin{Shaded}
\begin{Highlighting}[]
\FunctionTok{head}\NormalTok{(}\StringTok{"Levensverwachting.csv"}\NormalTok{)}
\end{Highlighting}
\end{Shaded}

\begin{verbatim}
## [1] "Levensverwachting.csv"
\end{verbatim}

This dataset provides us with information about the life expectancy
across different periods between 1996-2022.The numbers are provided on a
national and municipal level. The data at the municipal level are
calculated on the basis of a 4-year period.

\href{(https://statline.rivm.nl/portal.html?_la=nl&_catalog=RIVM&tableId=50132NED&_theme=101)}{URL
0f the data download page} + Metadata

\begin{Shaded}
\begin{Highlighting}[]
\FunctionTok{head}\NormalTok{(}\StringTok{"Ervarengezondheid\_Wijk\&Buurt.csv"}\NormalTok{)}
\end{Highlighting}
\end{Shaded}

\begin{verbatim}
## [1] "Ervarengezondheid_Wijk&Buurt.csv"
\end{verbatim}

The table contains self-reported health indicators through a
questionnaire about health, social situation, and lifestyle at
neighborhood, district, and municipal levels. The average of all people
living in a municipality or neighbourhood is computed.

\href{https://statline.rivm.nl/portal.html?_la=nl&_catalog=RIVM&tableId=50120NED&_theme=94}{URL
of the data download page} + Metadata

\begin{Shaded}
\begin{Highlighting}[]
\FunctionTok{head}\NormalTok{(}\StringTok{"Inkomen\_gemeente.csv"}\NormalTok{)}
\end{Highlighting}
\end{Shaded}

\begin{verbatim}
## [1] "Inkomen_gemeente.csv"
\end{verbatim}

For each household within a municipality or neighbourhood, the total
household income was calculated and then divided by the number of
household members. This results in the average income per person, which
accounts for differences in household size and provides a more accurate
measure of individual economic well-being.

\href{https://www.cbs.nl/nl-nl/maatwerk/2023/35/inkomen-per-gemeente-en-wijk-2020}{URL
of the data download page} + Metadata

\begin{Shaded}
\begin{Highlighting}[]
\NormalTok{inline\_code }\OtherTok{=} \ConstantTok{TRUE}
\end{Highlighting}
\end{Shaded}

These are things that are usually included in the metadata of the
dataset. For your project, you need to provide us with the information
from your metadata that we need to understand your dataset of choice.

\subsection{2.3 Describe the type of variables
included}\label{describe-the-type-of-variables-included}

Our data sets come from the CBS, an independent administrative body of
the Dutch government. CBS is responsible for providing the public with
unbiased, statistical information. Their data sets have a wide sample
pool which ensure reliability and relevance for our project.

Our sources provide us with data about life expectancy, experienced
health (ervaren gezondheid), and income at national, municipal, and
neighbourhood levels. These datasets complement each other well because
of their recent publication dates (2020+), making them very suitable for
integrating them in our analysis. they also provide us with information
about the health and income/wealth of Dutch citizens on different
levels, allowing us to assign one indicator of health (like experienced
health or life expectancy), with an income indicator (like income or net
worth). Someone's health or wealth is very difficult to quantify,
because there are many different factors at play. Why we decided on
these datasets is because we think that they provide us with the most
relevant variables when wanting to measure those things.

Despite their recent publication, they are still collected in different
years. Large global or national events (like COVID-19) may have affected
the data. These events could impact our data, skewing our results. Also
life expectancy and Experienced health are not perfect indicators for
ones health. For example, life expectancy is heavily impacted by one's
genetics, so income might have little influence in this. The life
expectancy could also vary due to accidents. If one group of people get
in fatal accidents more often, than this could also impact their life
expectancy. experienced health also has an obvious flaw, they are based
of reports that are not made by an export. That means that the values
gives by participants is inherently inaccurate.

\emph{For the sake of this example, I will continue with the
assignment\ldots{}}

\section{Part 3 - Quantifying}\label{part-3---quantifying}

\subsection{3.1 Data cleaning}\label{data-cleaning}

Say we want to include only larger distances (above 2) in our dataset,
we can filter for this.

\begin{Shaded}
\begin{Highlighting}[]
\CommentTok{\#mean(dataset$percollege)}
\end{Highlighting}
\end{Shaded}

Please use a separate `R block' of code for each type of cleaning. So,
e.g.~one for missing values, a new one for removing unnecessary
variables etc.

\subsection{3.2 Generate necessary
variables}\label{generate-necessary-variables}

Variable 1

We decided to make a new variable called ``correlation'' that we added

Variable 2

\subsection{3.3 Visualize temporal
variation}\label{visualize-temporal-variation}

\subsection{3.4 Visualize spatial
variation}\label{visualize-spatial-variation}

Here you provide a description of why the plot above is relevant to your
specific social problem.

\subsection{3.5 Visualize sub-population
variation}\label{visualize-sub-population-variation}

What is the poverty rate by state?

Here you provide a description of why the plot above is relevant to your
specific social problem.

\subsection{3.6 Event analysis}\label{event-analysis}

Analyze the relationship between two variables.

Here you provide a description of why the plot above is relevant to your
specific social problem.

\section{Part 4 - Discussion}\label{part-4---discussion}

\subsection{4.1 Discuss your findings}\label{discuss-your-findings}

\section{Part 5 - Reproducibility}\label{part-5---reproducibility}

\subsection{5.1 Github repository link}\label{github-repository-link}

\url{https://github.com/TijsReijling/Programmeren}

\subsection{5.2 Reference list}\label{reference-list}

Data set sources include links to their corresponding metadata.

\#Bovenaan welvaartsladder bijna 25 jaar langer in goede gezondheid.
(2022, December 20). Centraal Bureau Voor De Statistiek.
\url{https://www.cbs.nl/nl-nl/nieuws/2022/51/bovenaan-welvaartsladder-bijna-25-jaar-langer-in-goede-gezondheid}?

\#Pharos. (2022, July). Sociaal economische Gezondheidsverschillen
(SEGV).
\url{https://www.pharos.nl/factsheets/sociaaleconomische-gezondheidsverschillen-segv/}

\#Sociaaleconomische gezondheidsverschillen \textbar{} Volksgezondheid
en Zorg. (n.d.).
\url{https://www.vzinfo.nl/sociaaleconomische-gezondheidsverschillen}

Gezonde levensverwachting; inkomen en welvaart. (2025, May 23).{[}Data
set{]}. CBS dataportaal.
\url{https://opendata.cbs.nl/statline/portal.html?_la=nl&_catalog=CBS&tableId=85445NED&_theme=154}

Welzijn; kerncijfers, persoonskenmerken. (2025, March 20).{[}Data
set{]}. CBS dataportaal.
\url{https://opendata.cbs.nl/statline/portal.html?_la=nl&_catalog=CBS&tableId=85542NED&_theme=178}

Inkomen per gemeente en wijk, 2020. (2023, September 1).{[}Data set{]}.
Centraal Bureau voor de Statisktiek.
\url{https://www.cbs.nl/nl-nl/maatwerk/2023/35/inkomen-per-gemeente-en-wijk-2020}
(metadata included in dataset)

Gezondheid per wijk en buurt; 2012/2016/2020/2022. (2024, December 09).
{[}Data set{]}. RIVM dataportaal.
\url{https://statline.rivm.nl/portal.html?_la=nl&_catalog=RIVM&tableId=50120NED&_theme=94}

Levensverwachting op de leeftijd 0 en 65 jaar; geslacht, regio
1996-2022. (2024, December 9). {[}Data set{]}. RIVM dataportaal.
\url{https://statline.rivm.nl/portal.html?_la=nl&_catalog=RIVM&tableId=50132NED&_theme=101}

\end{document}
